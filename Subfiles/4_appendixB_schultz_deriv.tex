\subsection{Measuring dispersive solid surface energy, $\gamma_{S}^{D}$}
	
Assuming Young's equation (Equation \ref{young_eqn}) can be applied to a liquid-liquid(bulk phase)-solid ($L_{1}-L_{2}-S$) system, and the relationship follows:
\begin{equation}
\label{youngs}
	\gamma_{SL_{2}} = \gamma_{L_{1}L_{2}}\cos\theta_{SL_{1}} + \gamma_{SL_{1}}
\end{equation}

In this way, \gamSV, which is the driving force of spreading the one-liquid-phase method that causes complete wetting on high-surface energy metal surfaces, is replaced by $ \gamma_{SL_{2}} $, where $\gamma_{SL_{2}} <$ \gamSV. Hence, the contact angle in this system is measurable. According to Fowkes\cite{Fowkes1964}, $\gamma_{SL_{1}}$ and $\gamma_{SL_{2}}$ are given by:
\begin{equation} 
\label{gSL1}
	\gamma_{SL_{1}} = \gamma_{S} + \gamma_{L_{1}} - 2(\gamma_{S}^{D}\gamma_{L_{1}}^{D}) - I_{SL_{1}}^{P}
\end{equation}
\begin{equation}
\label{gSL2}
	\gamma_{SL_{2}} = \gamma_{S} + \gamma_{L_{2}} - 2(\gamma_{S}^{D}\gamma_{L_{2}}^{D}) - I_{SL_{2}}^{P}
\end{equation} 
where $\gamma$ and $\gamma^{D}$ are the surface energy and its dispersive component, respectively, and $I_{SL_{1}}^{P}$ is a specific (nondispersive) interaction term that encompasses all interactions between the solid and the liquid (dipole-dipole, dipole-induced dipole, hydrogen bonds, $\pi$ bonds, etc.) except London dispersion interactions.

%TODO Define and understand dispersive vs. polar components of surface energy.
%TODO Define and understand London dispersion forces
Substituting Equations \ref{gSL1} and \ref{gSL2} into \ref{youngs}:

\begin{equation} 
\label{schultz1}
	\gamma_{SL_{1}}-\gamma_{SL_{2}}+\gamma_{L_{1}L_{2}}\cos\theta_{SL_{1}} = 2(\gamma_{S}^{D})^{1/2}  [(\gamma_{L_{1}}^{D})^{1/2}-(\gamma_{L_{2}}^{D})^{1/2}] + I_{SL_{1}}^{P} - I_{SL_{2}}^{P}
\end{equation}

$L_{1}$ will signify water, $L_{2}$ as \nalk[s], and $I_{SL_{2}}^{P}$ may be considered equal to zero because the surface free energy of \nalk[s] only consists of the London dispersion term. This is because \nalk[s] only contain C-C and C-H atoms connected by $\sigma$-bonds, with a generic formula of C$_{n}$H$_{2n+2}$. C and H have very similar electronegativities of $\chi_{C}=2.55$ and $\chi_{H}=2.20$, respectively. This shows that all bonds in \nalk[s] are non-polar, hence there are no polar interactions in \nalk[s]. Our final equation is now:

\begin{equation} 
\label{schultz2}
	\gamma_{W}-\gamma_{H}+\gamma_{WH}\cos\theta_{W} = 2(\gamma_{S}^{D}\gamma_{W}^{D})^{1/2}-(\gamma_{H}^{D})^{1/2}] + I_{SW}^{P} 
\end{equation} 

\subsection{Measuring polar solid surface energy, $\gamma_{S}^{P}$}

Beginning with Equation \ref{schultz1} in the previous section and solving for $ I_{SL_{2}}^{P} $, with liquid $L_1$ as water (subscript W):

\begin{equation}
	I_{SL_2}^{P} = \gamma_{SL_2}-\gamma_{WL_2}\cos\theta_{SW}-2(\gamma_{S}^{D}\gamma_{L_2}^{D})^{1/2} + C
\end{equation}
\begin{equation}
	C = I_{SW}^{P} + 2(\gamma_{S}^{D}\gamma_{W}^{D})^{1/2} - \gamma_W
\end{equation}

In the above two equations, the values of $ \gamma_W$, $ \gamma_{W}^{D} $, $ \gamma_{S}^{D} $, and $I_{SW}^{P}$ are available experimentally from the previous section. Therefore, measurements of $ \gamma_{L_2}$, $ \gamma_{L_2}^{D} $, $\theta_{SW}$, and $ \gamma_{L_2}^{D} $ lead to a calculation of the polar interaction $I_{SW}^{P}$. 





