% test comment
\documentclass[12pt,letterpaper]{report}

\usepackage[utf8]{inputenc}
\usepackage{preamble}	%a local package of packages used in this document
	% All packages from this preamble were moved to the main file. 




\title{
	{\textbf{Orientation-dependent Surface Energy Characterization of Rare-Earth-Free Magnetostrictive Alloys for Abnormal Grain Growth Modeling }}\\
	{\large University of Maryland, College Park}\\
}
\author{\makebox[.9\textwidth]{\textbf{Michael N. Van Order}\thanks{Funded by the \NSF SUSCHEM - Collaborative Research program (grant number: DMR-1310447)}}\\~\\
	%		Materials Science and Engineering\\
	%		University of Maryland
	\and Dr. Alison Flatau\\
	%		Aerospace Engineering\\
	%		Univeristy of Maryland
	\and Dr. Suok-Min Na\\
	%		Aerospace Engineering\\
	%		University of Maryland
}
\date{20 February 2017}

\doublespacing

\begin{document}	


\chapter*{Abstract}
Abstract goes here. Yeah. You will do this last. 

High sensitivity semiconductor, optical, and sensing devices require single crystal metals for their isotropy which allows for unique material properties. However they are expensive and difficult to produce. By using abnormal grain growth (AGG) techniques, our group can produce single-crystal-like materials that achieve ~90\% performance of true single-crystals at ~5\% of the cost. Fully understanding AGG mechanisms is crucial to the growth of high quality, cost effective alloy fabrication. Many advances have been made to understand key parameters of AGG, and we postulate that the final piece lies in understanding the surface energy of our alloys. While several surface energy measurement techniques have been developed for low-energy plastic surfaces, high-energy metal surfaces have largely been ignored due to the complexity of sample preparation and experimentation. This dissertation investigates three measurement techniques targeted for high-surface-energy iron-alloy crystal facets. \par
The first of these techniques, the gallium drop contact angle method, examined a droplet of liquid metal gallium resting on our metal sample. By recording shape of this droplet, a value for surface energy of targeted crystal orientations is acquired using our derived thermodynamically-based mathematical model. This study experimentally confirmed trends that are predicted in theoretical models, but identified that oxide formation on the sample surface interferes with acquisition of accurate quantitative results. This revelation led to a more robust study that expands on classic drop shape analysis techniques and eliminates complications associated with oxide layer formation. I performed and analyzed multiple oxide removal procedures using X-ray photoelectron spectroscopy. The most promising procedures are polishing in an inert atmosphere and ion bombardment cleaning. Immersing the sample in an oil environment isolates this unstable iron-alloy surface from air and prevents oxidation. While in this environment, samples are probed with a deionized water droplet and a shape analysis is performed to calculate surface energy values using the Schultz method. I modified this method by utilizing a technique previous only used on plastics to prevent water from spreading. I hypothesize that patterning sample surfaces with an ion mill will stabilize droplets during shape measurements, thus generating reliable surface energy calculations. Success of this experiment could allow metallurgists to finally experimentally measure surface energy for any metal surface, thus providing confirmations of theory and sparking new ideas of how grain growth in metals can be controlled and even manipulated. \par 
A study suggested by my committee to expand the scope of my dissertation utilizes the micro-features of my patterned samples from the previous experiment growing equilibrium crystals at high temperatures. This seldom observed phenomenon can identify the temperatures at which specific metal crystals will grow and allow for a fundamental surface energy characterization. This complex and readily available technique will provide completeness to my dissertation and provide a basis for future researchers. This study will ultimately bolster my dissertation for the betterment of the surface science community as well as my own group’s goals of perfecting AGG. 



\begin{titlepage}
	\clearpage 
	\maketitle
	
	\thispagestyle{empty}
\end{titlepage}

\chapter*{Copyright Statement}
I declare that...

\chapter*{Dedication}
To mom, dad, sister, Alyson, and cats.



\chapter*{Acknowledgments}
I want to thank...

\tableofcontents
\listoffigures
\listoftables

\chapter{Introduction}
\section{Background}\label{section1}
%\subsection{Magnetostriction} %without * gives numbered section

\begin{outline}[enumerate]
\1 Magnetostriction
	\2 History and Discovery
		\3 Joule effect
		\3 Villari Effect
	\2 Materials
		\3 Terfenol-D
		\3 Galfenol
		\3 Alfenol
	\2 Uses and Applications
		\3 Sensors
			\4 Whiskers
			\4 Non-contact torque sensors
\1 Galfenol
	\2 Manufacturing 
		\3 Single crystal 
			\4 Arc melting
			\4 Directional solidification
		\3 Polycrystalline rolled thin sheets
			\4 Rolling (hot, warm, cold)
			\4 Atmospheric annealing 
				 Argon\\
				 Sulfur\\
			\4 Benefits
				 Rolling is much cheaper\\
				 Can grow grains to 90\% of sample surface
			\4 Restrictions
\1 Abnormal Grain Growth (AGG)
	\2 Theory
	\2 Proposed Mechanisms
	\2 Latest Research for AGG in Fe-based alloys
	\2 Need for Orientation-dependent Surface Analysis/Surface Energy Measurements
		\3 Latest Research on FeGa Surface Energy 
\1 Methods of Surface Energy Measurement Techniques
	\2 Contact Angles
		\3 Low Energy Surfaces
			\4 Owens-Wendt Method
		\3 High Energy Surfaces
		\3 
\end{outline}

\input{Subfiles/1_Background-Motivations}

\newpage
\chapter{Chapter 2 Title}
\section{Metal Surface Energy Measurement}\label{section2}

\begin{outline}[enumerate]
\1 See MRS Advances paper
	
\end{outline}

\input{Subfiles/2_completed_research}


\chapter{Chapter 3 Title}

\begin{outline}[enumerate]
\1 Schultz Method (Two-liquid-phase method)
	\2 Experimental Apparatus
\1 Nearly-flat surface results
	\2 Polishing procedure to remove oxide layer
		\3 Polishing and cleaning options at UMD
			\4 XPS results + quantitative analysis
		\3 Polishing at NSWC Carderock
	\2 Results for Carderock polished single crystal samples
		\3 Retained presence of iron oxide
	\2 Dry polish results (Week of Jan 2nd)
	
	\2 Plasma sputtered result (Week of Jan 9)
		\3 Using sputterer in Phaneuf glovebox
		
\1 Patterned surface results
	\2 Learn to use sputterer in Phaneuf glovebox?
	\2 Learn to do spin-coating from FabLab
		\3 Thick photoresist using A2 4620 (suggested by FabLab), which can get a 14 $\mu$m thick layer
	\2 Perform dry etch and spin coating in Phaneuf glovebox
	\2 Do UV processing somewhere
		\3 FabLab?
	\2 Use Ion mill for etching by scheduling with Xiaohang on Google Calendar
		\3 Ask for one more demo with him supervising. 
	
\1 Force curve measurement
	
	
	
\end{outline}


\input{Subfiles/3_proposed_research}

\chapter{Conclusion}
In conclusion...

\newpage
\appendix
\chapter{Derivation of Young's Equation}\label{appendixA}
\input{Subfiles/4_appendixA_young_deriv}

\chapter{Derivation of Schultz's two-liquid-phase method}\label{appendixB}
\input{Subfiles/4_appendixB_schultz_deriv}

\newpage
\bibliographystyle{unsrt}
\bibliography{BibTeX/library}


\end{document}