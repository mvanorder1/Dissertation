% test comment
\documentclass[12pt,letterpaper]{report}

\usepackage[utf8]{inputenc}
\usepackage{preamble}	%a local package of packages used in this document
	% All packages from this preamble were moved to the main file. 




\title{
	{\textbf{Orientation-dependent Surface Energy Characterization of Rare-Earth-Free Magnetostrictive Alloys for Abnormal Grain Growth Modeling }}\\
	{\large University of Maryland, College Park}\\
}
\author{\makebox[.9\textwidth]{\textbf{Michael N. Van Order}\thanks{Funded by the \NSF SUSCHEM - Collaborative Research program (grant number: DMR-1310447)}}\\~\\
	%		Materials Science and Engineering\\
	%		University of Maryland
	\and Dr. Alison Flatau\\
	%		Aerospace Engineering\\
	%		Univeristy of Maryland
	\and Dr. Suok-Min Na\\
	%		Aerospace Engineering\\
	%		University of Maryland
}
\date{20 February 2017}

\doublespacing

\begin{document}	


\chapter*{Abstract}
Abstract goes here. Yeah. You will do this last. 

High sensitivity semiconductor, optical, and sensing devices require single crystal metals for their isotropy which allows for unique material properties. However they are expensive and difficult to produce. By using abnormal grain growth (AGG) techniques, our group can produce single-crystal-like materials that achieve ~90\% performance of true single-crystals at ~5\% of the cost. Fully understanding AGG mechanisms is crucial to the growth of high quality, cost effective alloy fabrication. Many advances have been made to understand key parameters of AGG, and we postulate that the final piece lies in understanding the surface energy of our alloys. While several surface energy measurement techniques have been developed for low-energy plastic surfaces, high-energy metal surfaces have largely been ignored due to the complexity of sample preparation and experimentation. This dissertation investigates three measurement techniques targeted for high-surface-energy iron-alloy crystal facets. \par
The first of these techniques, the gallium drop contact angle method, examined a droplet of liquid metal gallium resting on our metal sample. By recording shape of this droplet, a value for surface energy of targeted crystal orientations is acquired using our derived thermodynamically-based mathematical model. This study experimentally confirmed trends that are predicted in theoretical models, but identified that oxide formation on the sample surface interferes with acquisition of accurate quantitative results. This revelation led to a more robust study that expands on classic drop shape analysis techniques and eliminates complications associated with oxide layer formation. I performed and analyzed multiple oxide removal procedures using X-ray photoelectron spectroscopy. The most promising procedures are polishing in an inert atmosphere and ion bombardment cleaning. Immersing the sample in an oil environment isolates this unstable iron-alloy surface from air and prevents oxidation. While in this environment, samples are probed with a deionized water droplet and a shape analysis is performed to calculate surface energy values using the Schultz method. I modified this method by utilizing a technique previous only used on plastics to prevent water from spreading. I hypothesize that patterning sample surfaces with an ion mill will stabilize droplets during shape measurements, thus generating reliable surface energy calculations. Success of this experiment could allow metallurgists to finally experimentally measure surface energy for any metal surface, thus providing confirmations of theory and sparking new ideas of how grain growth in metals can be controlled and even manipulated. \par 
A study suggested by my committee to expand the scope of my dissertation utilizes the micro-features of my patterned samples from the previous experiment growing equilibrium crystals at high temperatures. This seldom observed phenomenon can identify the temperatures at which specific metal crystals will grow and allow for a fundamental surface energy characterization. This complex and readily available technique will provide completeness to my dissertation and provide a basis for future researchers. This study will ultimately bolster my dissertation for the betterment of the surface science community as well as my own group’s goals of perfecting AGG. 



\begin{titlepage}
	\clearpage 
	\maketitle
	
	\thispagestyle{empty}
\end{titlepage}

\chapter*{Copyright Statement}
I declare that...

\chapter*{Dedication}
To mom, dad, sister, Alyson, and cats.



\chapter*{Acknowledgments}
I want to thank...mom, dad, sister, Alyson. 

Also to James Willems, Elyse Willems, Bruce Greene, Lawrence Sonntag, Adam Kovic, and Matt Peake, in no particular order, from Funhaus, for encouraging this stranger on the internet to do his best and be diligent with his work. Your work ethic is truly something I try to emulate everyday. 

\tableofcontents
\listoffigures
\listoftables

\chapter{Introduction}\label{chapter1}
\section{Background}\label{section1}
%\subsection{Magnetostriction} %without * gives numbered section

\begin{outline}[enumerate]
\1 Magnetostriction
	\2 History and Discovery
		\3 Joule effect
		\3 Villari Effect
	\2 Materials
		\3 Terfenol-D
		\3 Galfenol
		\3 Alfenol
	\2 Uses and Applications
		\3 Sensors
			\4 Whiskers
			\4 Non-contact torque sensors
\1 Galfenol
	\2 Manufacturing 
		\3 Single crystal 
			\4 Arc melting
			\4 Directional solidification
		\3 Polycrystalline rolled thin sheets
			\4 Rolling (hot, warm, cold)
			\4 Atmospheric annealing 
				 Argon\\
				 Sulfur\\
			\4 Benefits
				 Rolling is much cheaper\\
				 Can grow grains to 90\% of sample surface
			\4 Restrictions
\1 Abnormal Grain Growth (AGG)
	\2 Theory
	\2 Proposed Mechanisms
	\2 Latest Research for AGG in Fe-based alloys
	\2 Need for Orientation-dependent Surface Analysis/Surface Energy Measurements
		\3 Latest Research on FeGa Surface Energy 
\1 Surface Energy
	\2 Surface Energy vs. Surface Tension
		\3 solids vs. liquids
		\3 no real difference, same unites but different nomenclature
\1 Methods of Surface Energy Measurement Techniques
	\2 Contact Angles
		\3 Low Energy Surfaces
			\4 Owens-Wendt Method
		\3 High Energy Surfaces
			\4 Cleavage energy\cite{Gilman1960}
			\4 Atomic force curve measurement\cite{Drelich2004}
			\4 Schultz method\cite{Schultz1977,Schultz1977a,Schultz1992}
		\3 
\end{outline}

\input{Subfiles/1_Background-Motivations}

The interactions of water with a low-energy and high-energy solid surface differ significantly. The relative energy of a solid compared to a liquid has to do with the bulk nature of the solid. Solids with weak molecular crystals (e.g., fluorocarbons, hydrocarbons, etc.) where the molecules are held together by weak physical forces (e.g., van der Waals and hydrogen bonds) are termed low-energy solids. A very low input of energy is required to break these solids, thus they typically have surface energy values $<$72 mJ/m$^2$.  Metals, glasses, and ceramics are known as "hard solids" because the chemical bonds that hold them together (e.g., covalent, ionic, or metallic) are very strong. Thus, these surfaces are high-energy because of the high amount of energy needed to break these solids to make two new surfaces, (e.g. see Equation \ref{SFE} in Section \ref{define-surf-energy}). High energy solids typically have surfaces energies $>$72 mJ/m$^2$.  The threshold between these two types of surfaces lies in their surface energy values relative to the surface tension of water, $\sim$72 mJ/m$^2$. Water tends to partially wet a low energy surface and completely wet a high energy surface based on how much lesser or greater the surface energy of the solid is compared to water, respectively. 

I will investigate two approaches to overcome the challenges involved in measuring the surface energy of metals using contact angle measurements in the next two chapters. The first will study temperature variations of Galfenol surface energy using a droplet of gallium as the probe liquid. Gallium is a liquid at room temperature with a surface tension that is roughly one order of magnitude higher than that of water: $\gamma_{Ga}\sim$715.3 mJ/m$^2$. It is expected that a measurable contact angle will form between Galfenol and liquid gallium, and this information will be incorporated into an analytical model to extract the solid surface energy. The second will build on the two-liquid-phase technique to form a measurable water contact angle on the surface of Galfenol. To supplement this measurement, Galfenol samples will be patterned with a planned surface geometry to increase the water contact angle and properly determine the Young's contact angle to be used in surface energy calculations. Both experiments will use highly Goss textured Galfenol as well as single-crystal Galfenol samples to assure isotropic crystal orientation for contact angle measurements. 

Once experimental results are validated with published values, DFT predictions of the surface energy associated with different crystallographic orientations in Fe-Ga will be tested and verified. This task and the resultant capability should have broad applicability beyond the needs of this dissertation for extension to polycrystalline textured metals and thin films.
%This method has the potential to become a valuable tool for obtaining empirical surface energy data associated with AGG of Galfenol grains as well as other metallurgical studies.
%/research that lack experimental verification of surface energy around room temperature. 
%


\newpage
\chapter{Gallium Drop Contact Angle Experiment}\label{chapter2}


This chapter will focus on the gallium drop contact angle experiment. Its theory, development, results, and thoughts on improvement in future studies. 





\section{Theory behind model}\label{section2-1}

\subsection{thesis proposal}

\begin{figure}
	\centering
	\begin{subfigure}[b]{0.7\textwidth}
		\includegraphics[width=\textwidth,trim={0 0 0 2cm}]{youngs-ga}
	%	\caption{Interfacial tensions on Ga drop on solid surface}
		\label{fig:youngs-ga}
	\end{subfigure}
	%add desired spacing between images, e. g. ~, \quad, \qquad, \hfill etc. 
	%(or a blank line to force the subfigure onto a new line)
	\begin{subfigure}[b]{0.7\textwidth}
		\includegraphics[width=\textwidth,trim={0 2cm 0 0}]{thermal-expand-drop}
	%	\caption{Thermal expansion of drop}
		\label{fig:thermal-expand-drop}
	\end{subfigure}
	\caption{Schematics indicating notation used and contact angles $\theta_{i}$, radii of drop-solid contact area $r_{i}$, and radii of curvature for a spherical drop $R_{i}$, for two temperatures as thermal expansion induced tension $P$ strains the substrate.}
	\label{fig:therm-exp-ga}
\end{figure}
The gallium drop contact angle technique uses classic contact angle measurement techniques and drop size analysis on a liquid metal, gallium, resting on a metal surface as the metal is heated. Figure \ref{fig:therm-exp-ga} illustrates the experiment. Gallium was chosen as the probe liquid for our high surface energy metals, \gamSV $\sim$2000 mJ/m$^2$, because it is a liquid above 29.8\degree C and it has an average surface tension of $\gamma_{Ga}\sim$715.3 mJ/m$^2$.\cite{Hardy1985} This means that liquid Ga will not completely wet a bare metal surface as water does. Water has a relatively low surface tension, $\gamma_{water}\sim$72.0 mJ/m$^2$, compared to a solid metal surface energy, hence when interacting with a bare metal surface the water contact angle will reduce to zero, and no solid surface energy can be calculated, which can be seen in Equation \ref{young_eqn}. The surface tension of Ga varies by 5 mJ/m$^2$ in the temperature range that this experiment will be carried out, $\sim$30-100\degree C, and will be accounted for in the final surface energy calculation. Ga surface tension can be linear plotted as $\gamma_{Ga} = 709 - 0.066(T-29.8)$, where $T$ is the temperature and 29.8 is the melting point of Ga.\cite{Hardy1985} A plot of Ga surface tension vs. temperature from Hardy \etal is shown in Figure \ref{fig:se-vs-temp-ga}. 

%%%%%%%%%%%%%%%%%%%
%This paragraph explains the choice of gallium as a probe liquid in a contact angle experiment for a metal surface instead of water. 
%It establishes Ga as a main contributor to the experiment. One that must be characterized well. 
%%%%%%%%%%%%%%%%%%%


\begin{figure}
	\centering
	\includegraphics[width=0.5\linewidth,trim={0 0 0 1cm}]{se-vs-temp-ga}
	\caption{This figure plots the surface tension of pure Ga as temperature increases. This figure is from Hardy \etal \cite{Hardy1985}.}
	\label{fig:se-vs-temp-ga}
\end{figure}


Surface energy of FeGa will be calculated at temperatures ranging from 30-100\degree C in 10\degree C intervals. Changes in surface energy are introduced by thermal expansion of the substrate. By expanding the distance between atoms at the surface of the solid, the solid surface energy decreases. Intuitively, this makes sense because as the metallic bond distance increases, the less energy it would take to break those bonds. Electrons will begin to localize around the nuclei, as the attractive forces between atoms will decrease with the increased atomic separation. Ultimately, it will become easier for the metal to cleave and form two new surfaces, hence the surface energy decreases as temperature increases. 

Since a heated sample thermally expands, we can say that there is a tensile force on the edges of the stationary droplet of Ga. The triple point at every point of the droplet contact radius expands outward due to the thermal expansion of the probed surface. This tensile load can be called $ +P(T) $, seen in Figure \ref{fig:therm-exp-ga}, which effectively appears as a uniform radial force of the planar solid surface as temperature increases. By slowly heating the system by $\sim$1\degree C/min, the substrate expands uniformly while equilibrating at each temperature. Hence, terms due to variation in thermal energy and variation of the total Gibbs free energy are negligible. 

The surface energy of the substrate as a function of temperature, $\gamma_{SV}(T)$, can be related to $ +P(T) $ through the changes in the liquid Ga drop contact angle $\theta$, the radius $ r $ of the liquid gallium drop in contact with the metal surface, and the height $ h $ of the droplet hemisphere. A video system is used to precisely quantify dimension changes in substrate and liquid metal drop during thermal expansion.
%%%%%%%%%
%this paragraph introduces the model of finding solid surface energy from the gallium drop experiment. 
%
%As temperature is increased, a force is exerted on the triple line of the gallium sessile drop by the thermal expansion of the substrate.
%%%%%%%%%

The interfacial energy between the substrate solid and the liquid gallium drop, \gamSL, can be written as \gamSL$=P(T)/2\pi r$. The uniform tension introduced by thermal expansion of substrate can be expressed as:
\begin{equation}\label{uniform-tension}
	P(T) = E_{sub}\alpha_{sub}(T-T_{mp})
\end{equation}
where $T_{mp}$ is the melting temperature of gallium, $E$  is the substrate Young’s modulus, and $\alpha$ is the linear thermal expansion coefficient. A linear function can be used to write the gallium-air interfacial tension a function of temperature, \gamLV $= a-b(T-T_{mp})$, where \textit{a} and \textit{b} are positive constants found experimentally.\cite{Hardy1985,Alchagirov2005} Putting the terms for \gamSL and \gamSV into \hyperlink{youngeqn}{Young’s equation}\cite{Rudawska2009,Tadmor2004}:
\begin{equation*}%\label{youngs-eqn-ga1}
	\gamma_{SV} =  \frac{E_{sub}\alpha_{sub}(T-T_{mp})}{2\pi r} + \left[a-b(T-T_{mp})\right]\cos\theta
\end{equation*}
%%%%%%%%%%%%%%%%%%%%%
%Fully derives the gallium drop surface energy equation.
%This paragraph needs to be partially combined with the previous paragraph. 
%%%%%%%%%%%%%%%%%%%%%


A relationship between the variable radius $r(T)$ associated with the area of the circular region of solid-liquid contact $A_{SL}(T)$, the contact angle $\theta(T)$ and the volume $V$ of the spherical cap formed by the drop is derived next. To determine the radius $r(T)$, the geometric relationships based on the radius of curvature $R(T)$ of a sphere is mapped onto the hemispherical liquid cap. The drop is modeled as being part of a sphere whose radius is $R(T)$. It is assumed that he volume $V$ of the drop is the volume of the spherical cap, and remains constant. The radius $R(T)$ can then be expressed in terms of the volume and the angle:
\begin{equation*}\label{drop-geom}
	R(T) = V^{1/3} \left[\frac{\pi}{3} \left(2-3\cos\theta(T)+\cos^{3}\theta(T)\right)\right]^{-1/3}
\end{equation*}
Using the relation $r(T)=R(T)\sin\theta(T)$ (i.e. $\theta=$0\degree corresponds to complete wetting of the surface and at $\theta=90$\degree,  $r=R$) the following formula for surface energy as a function of temperature $T$ and contact angle $\theta(T)$ (shown as $\theta_{T}$) is:
\begin{equation}\label{youngs-eqn-ga}
	\gamma_{SV} =  \underbracket{\frac{E_{sub}\alpha_{sub}(T-T_{mp})}{2\pi}\left[\frac{\pi\left(2-3\cos\theta_{T}+\cos^{3}\theta_{T}\right)}{3V\sin^{3}\theta_{T}} \right]^{1/3}}_{\text{\gamSL(T)}} + \underbracket{\left[a-b(T-T_{mp})\right]}_{\text{\gamLV(T)}} \cos\theta_{T}
\end{equation}
This gives the surface energy \gamSV of specific grains as a function of temperature by measuring $T$ and $\theta$ at thermal equilibrium.





\section{Preliminary experiments}\label{section2-2}
In order to perform contact angle measurements on any material, the surface of that material must be pristine and well characterized. There can be no doubt that the probed surface is free of debris, organics, or oxides, for the case of metals, lest the contact angle be influenced by something other than the probed material. Proper sample storage and chemical cleaning (acetone $\Rightarrow$ methanol $ \Rightarrow $ DI rinse $ \Rightarrow $ N$_2$ dry) is satisfactory for removing particulates and organic residue, but exposing a bare metal surface by removing the native oxide layer is challenging. This chapter provides basic oxide removal techniques, but there is a more detailed oxide removal study in Chapter \ref{chapter3}. 
The surface must also be as close to atomically flat as possible. Most literature will approximate their surfaces as flat as long as the roughness is 10$^2$ greater than the contact area radius. 

One of the challenges in developing this new surface energy measurement capability is the need to expose the surface of the metal substrate that is shielded by surface oxides. The following strategies for removing the oxide layer and preventing  natural oxidation, thereby increasing the accuracy of measurement, will be employed independently and in combination:
\begin{outline}
	\1 A flux used in high-temperature metal joining processes plays roles of dissolving of the oxides on the metal surface and preventing of re-oxidation as a chemical agent.
	\1 Colloidal silica polishing with nano-sized particles, such as is used for precise surface observations, like EBSD scans, which require clean surfaces to accurately detect patterns.
	\1 Electro-polishing is effective for passivation of clean surfaces after chemical and mechanical polishing for removal of surface oxides. 
\end{outline}
%Results using the proposed approach were to be validated through comparison of results for oriented single crystals with published theoretical values for pure metal elements of a specific crystallography (e.g. Ni, Cu, Fe)  and comparison with experimental data from amorphous metals (e.g. Vitreloy or liquid steel) with destructive high temperature methods for measuring surface energy. 
It should be noted that all of these techniques will not prevent oxides from forming for an extended period of time. Therefore, the cleaning would have to be followed by isolation in high vacuum, an inert environment, or another liquid environment that prevents oxidation.

\subsection{Sample Preparation}
\subsubsection{Rolling and annealing}
Single crystal ingots were prepared by DOE Ames Laboratory, most likely using the Czochralski Method. Polycrystalline samples are prepared in the Aerosmart lab by Dr. Suok-Min Na using a progressive hot, warm, then cold rolling technique from a polycrystalline Galfenol ingot. The annealing process is described in \hyperlink{abnormal-grain-growth}{Chapter 1}. Single crystal samples are preferred for contact angle experiments because the prepared surface is isotropic, therefore multiple locations on a single sample surface can be probed and compared as equal surfaces. 


\subsubsection{Polishing}


Samples were prepared by manual grinding with incrementally higher grit SiC paper up to 1200 fine grit and then fine polishing using silica gel with 60 nm sized particles to minimize the roughness as well as the stressed surface states created during grinding.\cite{Hoffmann1987} Roughness measurements were performed on ten different 1$\mu$m x 1$\mu$m areas with an atomic force microscope (Veeco Dimension 3100) after fine polishing. Surface roughness values of ~1 nm were measured, thus there is minimal roughness influence on the contact angle measurements. Below are AFM images showing the difference between 1200 fine polishing and subsequent silica gel polishing. 
\begin{figure}
	\centering
	\includegraphics[width=\linewidth,trim={0 0 0 1cm}]{silica-polish}
	\caption{}
	\label{fig:silica-polish}
\end{figure}


%TODO insert an AFM image here to show roughness after a silica polish



\subsubsection{Electron Backscatter Diffraction}
%TODO Insert highly-textured polycrystal sample EBSD image made near beginning of project. 

%TODO single crystal EBSD images for (100), (110), and (111) samples that were vibratory polished
\subsubsection{Atomic Force Microscopy}
%TODO Expand on why AFM is a good enough calculator of roughness at a nanoscale. 

AFM is one of the simplest ways to determine roughness at the sub-nanometer scale, as opposed to profilometry which tends to lack the $<$100 nm resolution.

\subsection{Thermal Chamber Design Progression}
\subsubsection{Contact Angle Goniometer (Verison 1)}
Preliminary designs of my contact angle goniometer implement a radiative temperature control box which encloses an argon gas filled container where the sample resides, as seen in Figure \ref{fig:rad-temp-box}.  The presence of argon is meant to prevent any further oxidation of the Galfenol sample as well as the gallium droplet. The container was initially made of a clear acrylic plastic, but prolonged exposure to temperatures above 80\degree C caused thermal deformation of the plastic making longer experiments impossible to perform without environment contamination.  A clear pyrex container replaced the acrylic box to fix this issue.  Application of the liquid gallium drop to our surfaces was done via a mounted plastic syringe with disposable stainless steel hypodermic needles that were available at the time. Liquid gallium tends to adhere strongly to the stainless steel needle tips which makes wetting to the sample very difficult. The stainless steel tip tends to deform the highly viscous gallium drop resulting in non-uniform hemispheric drop shapes, as shown in Figure \ref{fig:deformed-ga}.
\begin{figure}[h]
	\centering
	\begin{subfigure}[c]{0.45\textwidth}
		\includegraphics[width=\linewidth]{rad-temp-box}
		\subcaption{~}
		\label{fig:rad-temp-box}		
	\end{subfigure}
	\begin{subfigure}[c]{0.45\textwidth} 
		\includegraphics[width=\linewidth]{deformed-ga}
		\subcaption{~}
		\label{fig:deformed-ga}		
	\end{subfigure}
	\caption{(a) The first design of our contact angle goniometer.  The acrylic container houses the argon environment and sample.  This design was modified with a more stable glass enclosure. (b) A highly deformed gallium drop next to the thermocouple on a ceramic YAG test sample at 45.5\degree C. The angle measured on this droplet was subtracted from 180\degree since it was not measured through the liquid.}
	\label{fig:prelim-design}
\end{figure}



%\begin{figure}
%	\centering
%		\includegraphics[width=\linewidth]{rad-temp-box}
%	\caption{(a) The first design of our contact angle goniometer.  The acrylic container houses the argon environment and sample.  This design was modified with a more stable glass enclosure. (b) A highly deformed gallium drop next to the thermocouple on a ceramic YAG test sample at 45.5\degree C.}
%	\label{fig:prelim-design}
%\end{figure}
The radiative box that housed this experiment had a high variability in temperature according to thermocouple readings, so a smaller apparatus with a top-side syringe opening is made to properly perform gallium drop tests at specific temperatures and prevent interaction with the experiment environment. There must also be bright white backlighting to obtain a high contrast drop profile. In this radiative box configuration, the high-reflecting liquid metal surface prevents a high contrast drop profile image, as seen in Figure \ref{fig:deformed-ga}. Proper contact angle measurements also require the gallium droplets to carefully wet the surface while forming an axisymmetric and spherical-like shape on the solid surface. A height adjustment system must be used to move the gallium pendant drop close enough to the sample surface for solid adhesive forces to overcome the adhesion to the needle. Lastly, the argon gas environment could be more well contained, instead of just filling up the glass sample enclosure from the bottom and spilling out the top due to higher density argon displacing the surrounding air environment. 

\subsubsection{Version 2}
\begin{figure}
	\centering
	\includegraphics[width=\linewidth,trim={0 0 0 1cm}]{enviro_chamber}
	\caption{The second version of our gallium contact angle goniometer. The aluminum enclosure conductively transfers heat, the gas lines flow Ar gas into the chamber, top-mounted thermocouples monitor the gas and sample temperature, and the glass windows allow for backlighting of the drop profile along with high resolution image capture using a DSLR camera.}
	\label{fig:enviro_chamber}
\end{figure}
The new experimental apparatus can be seen in Figure \ref{fig:enviro_chamber}. The main structure is made of aluminum with two round glass windows on the front and back. The aluminum is meant to conductively transfer heat to the substrate by means of a heating cable wrapped around the outside of the structure. The high thermal conductivity of aluminum allows for a quick transfer of heat, thus an increased control of sample temperature. The time percentage dial controller attached to the heating tape is calibrated with the sample temperature using a thermocouple placed on the sample surface. Sample temperature can be consistently controlled with $\pm$0.5\degree C accuracy. Backlighting greatly improved the drop profile contrast by having only one white light source coming from one side of the droplet, as seen in Figure \ref{fig:deformed_ga}. The Ar environment is also far more contained and controlled. The silicone sealant creates a nearly air-tight system where the Ar gas will displace all gas contaminants that could further oxidize the sample or gallium droplet. 
%These precautions are suitable for any metals samples tested because each sample is polished according to the procedure described above and then cleaned with acetone to remove any oxides. 
Once the sample is inserted and sealed in the environmental chamber, Ar gas prevents further oxidation throughout the experiment. A positive partial pressure is achieved in the chamber with an in- and out-valve to constant flow air out of the chamber.  
A high quality macro lens (Nikon AF Micro Nikkor 200-mm 1:4 D) is used to precisely quantify dimension changes in substrate and liquid metal drop during thermal expansion. 
\begin{figure}
	\centering
	\includegraphics[width=\linewidth,trim={0 0 0 1cm}]{DSLRcamera}
	\caption{This is a photograph of a picture being taken of a gallium droplet on highly-textured polycrystalline at 94.4\degree C (seen from thermocouple meter) using a Nikon DSLR camera with a long macro-lens. }
	\label{fig:DSLRcamera}
\end{figure}

\subsection{Gallium Oxide removal}


While extensive steps have been taken to inhibit oxidation on the metal surfaces, preventing oxidation on the surface of liquid gallium was a greater challenge. Pure gallium and Gallium-based alloy surfaces very quickly oxidize in ambient air environments, and form turning a thin layer of gallium oxide (Ga$_{2}$O$_{3}$ and Ga$_{2}$O).\cite{Regan1995,Regan1997,Scharmann2004} Like any other metal, Ga has a high potential for spontaneous oxidation, or passivation. This oxide layer is solid and remains elastic until it experiences a yield stress. Therefore, that an oxidized gallium droplet does not behave as a simple liquid, but as a viscoelastic material. In addition, the oxide layer of gallium is known to adhere to almost any solid surface, causing a severe stiction problem that interferes with interfacial energy measurements.\cite{Scharmann2004}. This shows how dramatically the gallium surface tension decreases when the oxide forms. Khan \etal showed that gallium oxide forms hydroxyl groups on their exterior surface, making the drops lyophilic as opposed to the expected lyophobic behavior of pure gallium, a high surface tension liquid.\cite{Hardy1985,Alchagirov2005} 

\begin{figure}
	\centering
	\begin{subfigure}[c]{0.45\textwidth}
		\includegraphics[width=\linewidth,trim={0 0 0 0}]{ga_41c}
		\subcaption{~}
		\label{fig:ga_41c}		
	\end{subfigure}
	\begin{subfigure}[c]{0.45\textwidth} 
		\includegraphics[width=\linewidth,trim={0 0 0 0}]{ga_100c}
		\subcaption{~}
		\label{fig:ga_100c}		
	\end{subfigure}
	\caption{Pure liquid gallium obtains viscoelastic properties when trace amounts of oxygen are present via formation of oxide shell. Non-axisymmetric Ga drops form on this iron substrate.}
	\label{fig:deformed_ga}
\end{figure}
Figure \ref{fig:deformed_ga} shows our direct observation of this phenomenon with teardrop shaped droplets formed by adhering to the iron surface while simultaneously being pulled upwards by the deposition needle. The general shape of these drops were unchanged for many hours even at temperatures approaching $\sim$100\degree C, thus exhibiting the stability of viscoelastic properties caused by the solid oxide layer. Removing the oxide layer from liquid gallium will return normal liquid properties to gallium and allow the use of axisymmetric drop analysis calculations: Young-Laplace equation, tangent method, and circle approximation method. 

Oxide removal permits liquid gallium to directly interact with metal surfaces instead of gallium oxide; the derived terms for \gamSL and \gamLV in Equation \ref{youngs-eqn-ga} become more robust. A number of techniques have been developed to remove and recover gallium oxide on liquid gallium: ultra-high vacuum (UHV) techniques,\cite{Regan1995,Regan1997} chemical vapor etching,\cite{Kim2013,Doudrick2014} and electrohydrodynamic phenomena.\cite{Khan2014} A chemical vapor etch is the best option for this experiment because it has a minimal effect on the surface of metals and the experimental apparatus does not need to be changed.  
To execute the vapor etch, a pendant drop of gallium was formed and a pipette of 37wt\% HCl was brought in close proximity to etch away the oxide layer. The same procedure was performed on the sessile drop of gallium on the desired surface to etch away any oxide left on top of the droplet, as seen in Figure \ref{fig:hcl_vapor_treat}. The contact angles of gallium on bare glass before and after HCl vapor treatment are similar to respective contact angle values in Kim \etal\cite{Kim2013}

The HCl vapor etch may even have the benefit of removing any native iron oxides (Fe$_2$O$_3$) from the FeGa surface itself since HCl is a known etchant of Fe$_2$O$_3$.\cite{Walker1991}



\begin{figure}
	\centering
	\includegraphics[width=\linewidth,trim={0 0 0 1cm}]{hcl_vapor_treat}
	\caption{This image shows the contact angle of gallium on a Galfenol sample in an argon environment before and after HCl vapor treatment.}
	\label{fig:hcl_vapor_treat}
\end{figure}







\section{Final experimental design and Results}\label{section2-3}

\subsubsection{Polycrystalline Metals}

With the HCl vapor treatment added to the procedure, temperature-varying gallium contact angle measurements in an argon environment were executed in the aluminum chamber on multiple metal substrates. Polycrystalline samples of high purity ($>$99.99\%) tin, copper, and iron were polished and had liquid gallium deposited on their surfaces. The temperature in the chamber was slowly ($<$10\degree C/min) increased from ~30\degree C to just below 100\degree C. Beyond 100\degree C, the heating tape becomes inconsistent with the intervals of heat applied to the chamber. Photographs of gallium drop profiles are taken at ~10\degree C intervals to observe progression of the drop shape as temperature increases. The same procedure is done as the temperature is decreased to 30\degree C to observe reversibility of this process, since we assume that the thermal expansion of the substrate drives the change in \gamSL from Equation \ref{youngs-eqn-ga}. 

A chemical vapor etch is used to replace the gallium oxide () layer with a gallium chloride layer, restoring the droplet surface tension to nearly that of pure Ga, as shown by Kim \etal To execute the vapor etch, a sessile drop of gallium was formed on the desired surface and a pipette of 37wt\% HCl was brought within 2 cm to etch away the gallium oxide layer. 

It is known that liquid gallium tends to corrode most metal surfaces.\cite{Lewandowski2015,Narh1998,Fitzgerald1999} Since experiments lasted for less than one hour, corrosion between the two metals should not be significant enough to effect the measurement. Later in this \hyperlink{xps-gallium}{section}, the surface stoichiometry effects of the liquid Ga droplet on Galfenol for varying temperatures is examined using X-ray photoelectron spectroscopy (XPS). For the polycrystalline tin and copper samples, gallium began to visibly corrode through the surface between 60\degree C and 70\degree C, as evident by a rapid contact angle decrease on only side of the drop profile, as seen in Figure \ref{fig:copper-tin-gallium-CA}. Polycrystalline iron samples did not experience corrosion problems throughout the experiments. It is expected that as the temperature increases, the thermal expansion of the solid will isotropically expand the drop thus decreasing the contact angle. Some gallium contact angles decrease on iron, but it is often that the decrease occurs on one side of the drop profile as the temperature increases. This is most likely due to the polycrystalline grains having different thermal expansions, hence the drop spreading is anisotropic. This suggests that a single-crystal grain or highly-textured surface grain is needed to properly observe an isotropic drop expansion. 
\begin{figure}
	\centering
	\includegraphics[width=\linewidth,trim={0 0 0 2cm}]{copper-tin-gallium-CA}
	\caption{These photographs show a gallium drop contact angle on polycrystalline Sn (left) and polycrystalline Cu (right), and the corrosive effects of the gallium on the same substrates.}
	\label{fig:copper-tin-gallium-CA}
\end{figure}



Since the iron sample did not corrode in the presence of gallium, we proceeded to a Galfenol sample with an abnormally grown \hkl(110) grain. Figure \ref{fig:ca_ebsd} shows the location of a gallium droplet in contact with the highly Goss-textured surface. Using the same temperature intervals, the right and left contact angles were measured and the surface energy
\begin{figure}
	\centering
	\includegraphics[width=\linewidth,trim={0 0 0 2cm}]{ca_ebsd}
	\caption{The location of a gallium drop on highly Goss-textured surface.}
	\label{fig:ca_ebsd}
\end{figure}
was calculated using Equation \ref{youngs-eqn-ga}. The contact angle measurements and surface energy calculations are shown in Figure \ref{fig:goss_se_msrmnt}. The contact angle measurements show anisotropic spreading behavior since the left contact angle recedes as temperature increases while the right contact angle advances. At 60.7\degree C, the contact angles reached close to the same value which may indicate an equilibrium point of combating thermal expansions caused by nearby island grains. 

Both right and left contact angles were measured at each temperature and used in Equation \ref{youngs-eqn-ga} to calculate the substrate surface energy. Plots of contact angle measurements and surface energy estimates are shown in Figure \ref{fig:goss_se_msrmnt}. The contact angle measurements show anisotropic spreading behavior since the left contact angle recedes as temperature increases while the right contact angle advances. Asymmetries may be associated with nearby island grains that are evident in the electron backscatter diffraction (EBSD) scan of Figure \ref{fig:ca_ebsd}. The magnitude of surface energy values in Figure \ref{fig:goss_ga_se} are on the order of 100 J/m$^2$ and the values increase as temperature increases. We had expected surface energy values to be closer in magnitude to DFT predicted values for $\alpha$-iron of $ \gamma_{SV}(T)$ = 2.0535 J/m$^2$, which has the same body-centered cubic crystal structure as Galfenol.\cite{Wang2000} We had also expected a decrease in surface energy with increasing temperature. This is because surface energy depends on the net inward cohesive force between atoms, and the cohesive force binding atoms to one another will decrease as a temperature increase causes atoms to vibrate more rapidly. We do not yet understand why the model leads to unexpected trends. We are currently examining the role of the $\gamma_{SL}(T)$ term in the model, as it is three orders of magnitude larger than the $\gamma_{SV}(T)$ term (due to the contribution from the Young's modulus value associated with Galfenol) and dominates the results.

\begin{figure}[h]
	\centering
	\begin{subfigure}[c]{0.47\textwidth}
		\includegraphics[width=\linewidth]{CAvsTemp_Goss-Galfenol3}
		\subcaption{~}
		\label{fig:goss_ga_ca}		
	\end{subfigure}
	\begin{subfigure}[c]{0.47\textwidth} 
		\includegraphics[width=\linewidth]{SEvsTemp_Goss-Galfenol3}
		\subcaption{~}
		\label{fig:goss_ga_se}		
	\end{subfigure}
	\caption{(a) Blue and red dots show left and right contact angle, respectively, of liquid gallium on \hkl(110) Fe-Ga. (b) Calculated surface energy values of \hkl(110) Fe-Ga grains using Equation \ref{youngs-eqn-ga}.}
	\label{fig:goss_se_msrmnt}
\end{figure}





\subsubsection{Single-crystal Galfenol}
Using the same acid vapor etch, Ga droplets were placed on samples of single crystal Galfenol. Contact angle values decrease from the (110), (100), and (111) facets (Figure 4), in that order, although the angles for the (100) and (111) facets are not statistically different (within one standard deviation of one another). According to Young’s equation, these contact angles should be inversely proportional to surface energy. The measurements indicate that the surface energy of a (110) surface has the lowest surface energy as expected based on DFT predictions.[5] The magnitudes of angles for (100) and (111) surfaces were statistically similar. Ga contact angles values on highly-textured Goss (110) grains and the (110) single crystal facet are within one standard deviation of one another, and thus in excellent agreement. 
%TODO insert single xtal figures here

\subsubsection{X-ray photoelectron spectroscopy study on Ga corrosion}

\hypertarget{xps-gallium}{XPS} measurements were performed after Ga drop experiments at the University of Maryland Surface Analysis Center using a high sensitivity Kratos AXIS 165 spectrometer to examine the effect of Ga droplets with a chloride shell affects the surface stoichiometry of Galfenol. Samples were prepared by placing a Ga drop on the Galfenol surface, heating the sample to a temperature of 30\degree C, 50\degree C ,70\degree C or 90\degree C, and holding that temperature for 15 minutes. The 70\degree C sample was damaged during XPS testing. XPS was also performed on a control sample that was not exposed to a Ga drop or elevated temperatures. 


XPS measurements show an increasing intensity in the Ga$_2$O$_3$ (1116.7 eV) peak in samples at elevated temperatures. This was expected as surface adhesion of Ga was likely greater in samples exposed to higher temperatures. Excess surface Ga would have oxidized in the time between the thermal tests and XPS due to exposure to ambient conditions. In the Fe region, a strong signal of Fe$_2$O$_3$ (710.8 eV) with a small peak of metallic Fe (706.7 eV) was present on the control sample. On the 30\degree C, 50\degree C and 90\degree C samples, the Fe$_2$O$_3$ peaks diminished and the metallic Fe peak became dominant. The increased presence of Ga at elevated temperatures suggests that the Ga may have etched away the Fe$_2$O$_3$ and exposed metallic Fe, as Ga is a corrosive element in the liquid state. The Ga signal is attributed solely to the Ga droplet as Fe has a higher potential for oxidation than Ga, shown by the heavy presence of Fe$_2$O$_3$ at the surface, and therefore Fe has a dominant concentration at the surface. 

%TODO insert figure of XPS measurements that show the above results. 
	
	

\section{Conclusions and next steps}\label{section2-4}
We have developed a new method for measuring the surface energy of targeted crystal orientations on metal surfaces. The model calculates surface energies on Goss-textured polycrystalline Galfenol and single crystal samples that are two orders of magnitude greater than theoretical predictions for $\alpha$-Fe. We have observed a qualitative trend in single crystal samples of different crystallographic orientations that matches predictions made using DFT simulations, where $\gamma_{100}$ and $\gamma_{111}$  are similar in magnitude ($<$5\% difference) and $\gamma_{110}$ is ~20\% less than values of $\gamma_{100}$ and $\gamma_{111}$. Future contact angle experiments will be performed strictly on single crystal samples due to greater uniformity of crystal orientations at the surface when compared to highly-textured rolled Galfenol sheets with multiple island grains at the surface. This uniformity will guarantee consistent interactions at the Ga-Galfenol-Ar triple line. The presence of a nanometer layer of Fe$_2$O$_3$ on Galfenol hinders this surface energy measurement. More thorough cleaning and polishing techniques, including chemical and plasma etches, in inert environments will be employed to eliminate or mitigate this layer formation. The model will be tested on large grains of oriented iron crystals and on well-characterized low energy surfaces (i.e. PTFE, etc.) with water as the probe liquid at temperatures below room temperature. We will revisit our derivation to address potential sources of error in magnitude. The interaction between the Ga and Galfenol interface is not well understood, and an in-situ measurement would need to be done to observe this reaction or lack thereof. Currently, this in-situ measurement is beyond current capabilities, so it is unknown how it would even affect the Ga contact angle measurement. Further investigation of this model and method offers potential for surface energy characterization of any metal with grain sizes $ > $2 mm to accommodate the minimum size of the probe droplet.

After presenting these preliminary findings at the XXIV International Materials Research Congress and 2015 MRS Fall Meeting,\cite{VanOrder2015a,VanOrder2015} I discussed possible avenues of improvement for the gallium contact angle experiment with colleagues and potential collaborators. A more concrete measurement of \gamSL between liquid gallium and solid would need to be examined, a very non-trivial task. Ultimately, we decided to suspend the gallium drop experiment and re-evaluate the measurement strategy. 
 %for obtaining orientation-dependent surface energies of magnetostrictive materials. 
 
Overall, this was an exploration of well-proposed and innovative research. Assumptions were made about how the system would interact and efforts to achieve good results in the lab were made. It was determined that these assumptions were too great and the experiment failed to meet expectations. Much was learned about the complexity of surface energy research from both a theoretical and experimental perspective. This failure persuaded me to find a more robust and promising method for accomplishing project goals within the constraints of orientation-dependent surface energy characterization. 

%TODO parse this section of article to LAtex code



%\section{Metal Surface Energy Measurement}\label{section2}
%
%\begin{outline}[enumerate]
%\1 See MRS Advances paper
%	
%\end{outline}

%\input{Subfiles/2_completed_research}


\chapter{Patterned Two-Liquid-Phase Method}\label{chapter3}

The gallium drop technique study experimentally confirmed trends that are predicted in theoretical models, but identified that oxide formation on the sample surface interferes with acquisition of accurate quantitative results. This revelation led to a more robust study that expands on a classic drop shape analysis technique, the two-liquid-phase method, and eliminates complications associated with oxide layer formation. By immersing the sample in an oil environment, Galfenol surfaces are isolated from air, thus preventing oxidation. While in this environment, samples are probed with a deionized water droplet and a shape analysis is performed to calculate surface energy values using the Schultz method. Due to the added pressure from an immiscible liquid, the water droplet will not completely wet the Galfenol surface as it would in air due to the very high surface energy. I modified this method by utilizing a technique previous only used on plastics to prevent water from spreading. I hypothesize that patterning sample surfaces with an ion mill will stabilize droplets during shape measurements, thus generating reliable surface energy calculations. I performed and analyzed multiple oxide removal procedures using X-ray photoelectron spectroscopy. The most promising procedures, polishing in an inert atmosphere and ion bombardment cleaning, show much greater surface wetting of Galfenol by water while immersed in a hydrocarbon environment.

Success of this experiment could allow metallurgists to finally experimentally measure surface energy for any metal surface, thus providing confirmations of theory and sparking new ideas of how grain growth in metals can be controlled and even manipulated. 



\begin{outline}[enumerate]
\1 Schultz Method (Two-liquid-phase method)
	The second will build on the two-liquid-phase technique to form a measurable water contact angle on the surface of Galfenol. To supplement this measurement, Galfenol samples will be patterned with a planned roughness to increase the water contact angle and properly determine the Young's contact angle to be used in surface energy calculations. Both experiments will use highly Goss textured Galfenol as well as single-crystal Galfenol samples to assure isotropic crystal orientation for contact angle measurements. 
	
	
	\2 Experimental Apparatus
	
	
\1 Nearly-flat surface results
	\2 Polishing procedure to remove oxide layer
		\3 Polishing and cleaning options at UMD
			\4 XPS results + quantitative analysis
		\3 Polishing at NSWC Carderock
	\2 Results for Carderock polished single crystal samples
		\3 Retained presence of iron oxide
	\2 Dry polish results (Week of Jan 2nd)
	
	\2 Plasma sputtered result (Week of Jan 9)
		\3 Using sputterer in Phaneuf glovebox
		
\1 Patterned surface results
	\2 Learn to use sputterer in Phaneuf glovebox?
	\2 Learn to do spin-coating from FabLab
		\3 Thick photoresist using A2 4620 (suggested by FabLab), which can get a 14 $\mu$m thick layer
	\2 Perform dry etch and spin coating in Phaneuf glovebox
	\2 Do UV processing somewhere
		\3 FabLab?
	\2 Use Ion mill for etching by scheduling with Xiaohang on Google Calendar
		\3 Ask for one more demo with him supervising. 
	
\1 Force curve measurement
	
	
	
\end{outline}


\input{Subfiles/3_proposed_research}

\chapter{Conclusion}
In conclusion...

\newpage
\appendix
\chapter{Derivation of Young's Equation}\label{appendixA}
\input{Subfiles/4_appendixA_young_deriv}

\chapter{Derivation of Schultz's two-liquid-phase method}\label{appendixB}
\input{Subfiles/4_appendixB_schultz_deriv}

\newpage
\bibliographystyle{unsrt}
\bibliography{Bibtex/library}


\end{document}