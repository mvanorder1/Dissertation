% test comment
\documentclass[12pt,letterpaper]{report}

\usepackage[utf8]{inputenc}
\usepackage{preamble}	%a local package of packages used in this document
	% All packages from this preamble were moved to the main file. 




\title{
	{\textbf{Orientation-dependent Surface Energy Characterization of Rare-Earth-Free Magnetostrictive Alloys}}\\
	{\large University of Maryland, College Park}\\
}
\author{\makebox[.9\textwidth]{\textbf{Michael N. Van Order}\thanks{Funded by the \NSF SUSCHEM - Collaborative Research program (grant number: DMR-1310447)}}\\~\\
	%		Materials Science and Engineering\\
	%		University of Maryland
	\and Dr. Alison Flatau\\
	%		Aerospace Engineering\\
	%		Univeristy of Maryland
	\and Dr. Suok-Min Na\\
	%		Aerospace Engineering\\
	%		University of Maryland
}
\date{13 December 2016}

\doublespacing

\begin{document}	


\chapter*{Abstract}
Abstract goes here. Yeah. You will do this last. 

\begin{titlepage}
	\clearpage 
	\maketitle
	
	\thispagestyle{empty}
\end{titlepage}
\chapter*{Copyright Statement}
I declare that...

\chapter*{Dedication}
To mom, dad, sister, Alyson, and cats.



\chapter*{Acknowledgments}
I want to thank...

\tableofcontents
\listoffigures
\listoftables

\chapter{Introduction}
\section{Background}\label{section1}
%\subsection{Magnetostriction} %without * gives numbered section

\begin{outline}[enumerate]
\1 Magnetostriction
	\2 History and Discovery
	\2 Materials
	\2 Uses and Applications
\1 Galfenol
	\2 Manufacturing 
	\2 Benefits
	\2 Restrictions
\1 Abnormal Grain Growth (AGG)
	\2 Theory
	\2 Proposed Mechanisms
	\2 Latest Research for AGG in Fe-based alloys
	\2 Need for Orientation-dependent Surface Analysis/Surface Energy Measurements
		\3 Latest Research on FeGa Surface Energy 
\1 Methods of Surface Energy Measurement Techniques
	\2 Contact Angles
		\3 Low Energy Surfaces
			\4 Owens-Wendt Method
		\3 High Energy Surfaces
		\3 
\end{outline}

\input{Subfiles/1_Background-Motivations}

\newpage
\chapter{Chapter 2 Title}
\section{Metal Surface Energy Measurement}\label{section2}

\begin{outline}[enumerate]
\1 See MRS Advances paper
	
\end{outline}

\input{Subfiles/2_completed_research}


\chapter{Chapter 3 Title}

\begin{outline}[enumerate]
\1 Schultz Method (Two-liquid-phase method)
	\2 Experimental Apparatus
\1 Nearly-flat surface results
	\2 Polishing procedure to remove oxide layer
		\3 Polishing and cleaning options at UMD
			\4 XPS results + quantitative analysis
		\3 Polishing at NSWC Carderock
	\2 Results for Carderock polished single crystal samples
		\3 Retained presence of iron oxide
	\2 Dry polish results (Week of Jan 2nd)
	
	\2 Plasma sputtered result (Week of Jan 9)
		\3 Using sputterer in Phaneuf glovebox
		
\1 Patterned surface results
	\2 Learn to use sputterer in Phaneuf glovebox?
	\2 Learn to do spin-coating from FabLab
		\3 Thick photoresist using A2 4620 (suggested by FabLab), which can get a 14 $\mu$m thick layer
	\2 Perform dry etch and spin coating in Phaneuf glovebox
	\2 Do UV processing somewhere
		\3 FabLab?
	\2 Use Ion mill for etching by scheduling with Xiaohang on Google Calendar
		\3 Ask for one more demo with him supervising. 
	
\1 Force curve measurement
	
	
	
\end{outline}


\input{Subfiles/3_proposed_research}

\chapter{Conclusion}
In conclusion...

\newpage
\appendix
\chapter{Derivation of Young's Equation}\label{appendixA}
\input{Subfiles/4_appendixA_young_deriv}

\chapter{Derivation of Schultz's two-liquid-phase method}\label{appendixB}
\input{Subfiles/4_appendixB_schultz_deriv}

\newpage
\bibliographystyle{unsrt}
\bibliography{BibTeX/library}


\end{document}