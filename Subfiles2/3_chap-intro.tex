TEST TERST 

The gallium drop technique study experimentally confirmed trends that are predicted in theoretical models, but identified that oxide formation on the sample surface interferes with acquisition of accurate quantitative results. This revelation led to a more robust study that expands on a classic drop shape analysis technique, the two-liquid-phase method, and eliminates complications associated with oxide layer formation. By immersing the sample in an oil environment, Galfenol surfaces are isolated from air, thus preventing oxidation. While in this environment, samples are probed with a deionized water droplet and a shape analysis is performed to calculate surface energy values using the Schultz method. Due to the added pressure from an immiscible liquid, the water droplet will not completely wet the Galfenol surface as it would in air due to the very high surface energy. I modified this method by utilizing a technique previous only used on plastics to prevent water from spreading. I hypothesize that patterning sample surfaces with an ion mill will stabilize droplets during shape measurements, thus generating reliable surface energy calculations. I performed and analyzed multiple oxide removal procedures using X-ray photoelectron spectroscopy. The most promising procedures, polishing in an inert atmosphere and ion bombardment cleaning, show much greater surface wetting of Galfenol by water while immersed in a hydrocarbon environment.

Success of this experiment could allow metallurgists to finally experimentally measure surface energy for any metal surface, thus providing confirmations of theory and sparking new ideas of how grain growth in metals can be controlled and even manipulated. 

