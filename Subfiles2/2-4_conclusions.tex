We have developed a new method for measuring the surface energy of targeted crystal orientations on metal surfaces. The model calculates surface energies on Goss-textured polycrystalline Galfenol and single crystal samples that are two orders of magnitude greater than theoretical predictions for $\alpha$-Fe. We have observed a qualitative trend in single crystal samples of different crystallographic orientations that matches predictions made using DFT simulations, where $\gamma_{100}$ and $\gamma_{111}$  are similar in magnitude ($<$5\% difference) and $\gamma_{110}$ is ~20\% less than values of $\gamma_{100}$ and $\gamma_{111}$. Future contact angle experiments will be performed strictly on single crystal samples due to greater uniformity of crystal orientations at the surface when compared to highly-textured rolled Galfenol sheets with multiple island grains at the surface. This uniformity will guarantee consistent interactions at the Ga-Galfenol-Ar triple line. The presence of a nanometer layer of Fe$_2$O$_3$ on Galfenol hinders this surface energy measurement. More thorough cleaning and polishing techniques, including chemical and plasma etches, in inert environments will be employed to eliminate or mitigate this layer formation. The model will be tested on large grains of oriented iron crystals and on well-characterized low energy surfaces (i.e. PTFE, etc.) with water as the probe liquid at temperatures below room temperature. We will revisit our derivation to address potential sources of error in magnitude. The interaction between the Ga and Galfenol interface is not well understood, and an in-situ measurement would need to be done to observe this reaction or lack thereof. Currently, this in-situ measurement is beyond current capabilities, so it is unknown how it would even affect the Ga contact angle measurement. Further investigation of this model and method offers potential for surface energy characterization of any metal with grain sizes $ > $2 mm to accommodate the minimum size of the probe droplet.

After presenting these preliminary findings at the XXIV International Materials Research Congress and 2015 MRS Fall Meeting,\cite{VanOrder2015a,VanOrder2015} I discussed possible avenues of improvement for the gallium contact angle experiment with colleagues and potential collaborators. A more concrete measurement of \gamSL between liquid gallium and solid would need to be examined, a very non-trivial task. Ultimately, we decided to suspend the gallium drop experiment and re-evaluate the measurement strategy. 
 %for obtaining orientation-dependent surface energies of magnetostrictive materials. 
 
Overall, this was an exploration of well-proposed and innovative research. Assumptions were made about how the system would interact and efforts to achieve good results in the lab were made. It was determined that these assumptions were too great and the experiment failed to meet expectations. Much was learned about the complexity of surface energy research from both a theoretical and experimental perspective. This failure persuaded me to find a more robust and promising method for accomplishing project goals within the constraints of orientation-dependent surface energy characterization. 

%TODO parse this section of article to LAtex code
